\documentclass[a4paper]{article}

% standard packages
\usepackage{amssymb}
\usepackage{euscript}
\usepackage{eurosym}
\usepackage{graphicx}
\graphicspath{{img/}}
\usepackage{color}
\usepackage{epsfig}
\usepackage{fullpage}
\usepackage[colorlinks=true]{hyperref}
\usepackage{titling}

% title style
\pretitle{\noindent\LARGE}
\posttitle{\\[1ex]}
\preauthor{\large}
\postauthor{,}
\predate{\large(}
\postdate{)}
\date{last update: \today}

% large page size
\oddsidemargin -1cm
\topmargin -1cm
\textwidth 18cm
\textheight 27cm
\pagestyle{empty}


% PDF figure (floating)
\newcommand{\image}[3]{
\begin{figure}[#1]
\begin{center}
\caption{\small#3}
\includegraphics{img_#2.pdf}
\label{image:#2}
\end{center}
\end{figure}
}

% simple image
\newcommand\simpleimage[3][\linewidth]{
\smallskip\begin{center}\vbox{\noindent
\includegraphics[width=#1]{#2}\\%
\parbox{#1}{\it#3}}\end{center}
\medskip}

%\newcommand\simpleimage[3][\linewidth]{
%\smallskip\vbox{\noindent
%\includegraphics[width=#1]{#2}\\%
%\it#3}
%\medskip}


% companies
\def\SchaefferAG{\href{https://www.schaeffer-ag.de/en/}{Schaeffer-AG}}
\def\MultiCB{\href{https://portal.multi-circuit-boards.eu}{Multi-CB}}
\def\BlueFors{\href{https://www.bluefors.com/}{BlueFors}}

\def\Ohmite{\href{https://www.ohmite.com}{Ohmite}}
\def\Firmetal{\href{http://www.firmetal.com}{Firmetal}}
\def\Bruker{\href{https://www.bruker.com}{Brucker}}

% use this to refer to Farnell and RS numbers:
\def\FarnellN#1{\href{https://fi.farnell.com/#1}{Farnell:#1}}
\def\RSN#1{\href{https://fi.rsdelivers.com/productlist/search?query=#1}{RS:#1}}

% products:
\def\OhmiteOFOD{\href{https://www.ohmite.com/assets/docs/res_od_of_oa.pdf}{Ohmite OF/OD series}}

% Supplementary materials on github <report> <file>
\def\GitFile#1#2{\href{https://github.com/slazav/he3notes/raw/master/#1/#2}{#1/#2}}

% Supplementary materials on my users.aalt.fi page
\def\WWWFile#1{\href{https://users.aalto.fi/~zavyalv1/#1}{#1}}


%progams
\def\MagnettiProg{\href{https://github.com/slazav/magneetti}{\tt magnetti}}



\title{Calibration of VTT pressure sensor}
\author{V.Zavjalov}

\twocolumn
\begin{document}
\maketitle

{\bf Experimantal setup.} For measurements we used Andeen-Hagerling AH
2500A capacitance bridge working at 1~kHz. The voltage~$U$ applied by the
bridge to the pressure sensor can be chosen in the range from 0.5~mV to
15~V RMS. At~$U>1$~V the noise of the measurement was about
$1\cdot10^{-3}$~pF, this corresponds to the pressure variations of about
$0.3$~mbar. Pressure was measured by Wallace \& Tiernan absolute pressure
gauge with working range $0\ldots1$~bar and resolution $\sim 0.3$~mbar.
%(For low pressure measurements we also use Agilent FRG-730 Pirani gauge.)
Temperature was measured by PT1000 platinum thermometer.

%%%%%%%%%%%%%%%%%%%%%%%%%%%%%%%%%%%%%%%%%%%%%%%%%%%%%%%%%%
\image{h}{pcal}{Fig.~1. (A) Capacitance of the pressure sensor as a
function of pressure. Measurements for three exctation voltages (red 15
V, green 7.5 V, blue 1.5 V) and two temperatures (circles 295~K, squares
77.2~K) are shown. Lines are fits to the model~(\ref{eq:mod}), made
separately for each temperature. Data for other voltage values (3~V,
0.75~V, 0.35~V, 0.25~V, 0.1~V) was used for fitting, but not shown on the
picture.
(B) Difference between measured data and
model~(\ref{eq:mod}). Same data as on (A) are presented. }
%%%%%%%%%%%%%%%%%%%%%%%%%%%%%%%%%%%%%%%%%%%%%%%%%%%%%%%%%%

{\bf Pressure dependence.} We found that at constant temperature the
capacitance~$C$ of the pressure sensor as a function of pressure~$P$ and
applied voltage~$U$ can be fit by a model with four parameters, $C_0$,
$C_1$, $P_0$, $U_0$:
\begin{equation}\label{eq:mod}
C = C_0 + C_1\ \frac{1}{d},
\end{equation}
where $d$ is determined by a non-linear equation:
\begin{equation}
d = 1 - \frac{P}{P_0} - \left(\frac{1}{d}\ \frac{U}{U_0}\right)^2.
\end{equation}
In this model the sensor is treated as a sum of a constant capacitor
$C_0$ and a flat capacitor $C_1$ with dimensionless distance~$d$ affected
linearly by applied force. One component of the force is proportional to
the pressure and another one is electrostatic force proportional to $U^2/d^2$. In
this model the capacitor is collapsed at $P=P_0$ or
$U=2/(3\sqrt{3})\,U_0\approx 0.385\,U_0$. We checked that above~$\approx 1.6$~bar
loss in the capacitor grows quickly and it becomes imposiible to measure
the capacitance.

Fig. 1 shows measured capacitance as a function of~$P$ and~$U$ at room
temperature (295 K) and temperature of liquid nitrogen (77.2~K). One can
see that model works well everywhere except the region of high pressures
and high excitation voltages (this corresponds to the smallest values of
distance~$d$). At small exitation voltages ($<3V$) the model gives us the
accuracy $\sim 10^{-3}$~pF comparable to the noise in the capacitance measurement.

Measured parameters for room temperature (295K), temperature of liquid
nitrogen (77.2 K) and temperature of liquid helium (4.2K) are in the
table:

\begin{tabular}{lll}
$T$, K & $C_0$, pF          & $C_1$, pF \\\hline
295.04 & $24.0821\pm0.0084$ & $9.0052\pm0.0082$\\
77.2   & $23.4230\pm0.0065$ & $8.8738\pm0.0064$\\
4.2    & $23.3702\pm0.0098$ & $8.8544\pm0.0096$\\
\hline
\end{tabular}

\begin{tabular}{llll}
$T$, K & $P_0$, bar & $U_0$, V RMS\\\hline
295.0 & $2.9864\pm0.0019$ & $178.854\pm0.691$\\
77.2  & $3.0082\pm0.0015$ & $179.846\pm0.530$\\
4.2   & $3.0043\pm0.0023$ & $171.643\pm0.713$\\
\hline
\end{tabular}
\medskip

%{\bf Temperature dependence.}
%Measurements similar to Fig.~1 are done at several temperatures giving us
%parameters $C_0$, $C_1$, $\alpha$ and~$\beta$. We also performed
%measurements of the capacitance during slow heating at 0 and 1 bar with
%various voltages~$U$.
%
%TODO


{\bf Helium diffusion.} In helium atmosphere pressure readings of the
sensor start to shift because of diffusion of helium inside the sensor
inner volume. In air or vacuum the diffusion goes in the opposite
direction, restoring the original readings. This effect exist only at
high temperatures. At room temperature we observed filling of the inner
volume with time constant $\sim2.5$~h. At temperature of liquid nitrogen
(77.2~K) we have not observed any visible change during 15~h
(See Fig~2.).

%%%%%%%%%%%%%%%%%%%%%%%%%%%%%%%%%%%%%%%%%%%%%%%%%%%%%%%%%%
\image{h}{hdiff}{Fig.~1. Helium diffusion. At room temperature
(295 K) helium diffuse into sensor's inner volume, filling it with time
constant $\sim 2.5$~h. At temperature of liquid nitrogen no diffusion have
been observed.}
%%%%%%%%%%%%%%%%%%%%%%%%%%%%%%%%%%%%%%%%%%%%%%%%%%%%%%%%%%


\end{document}
