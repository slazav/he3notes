\documentclass[a4paper]{article}
% standard packages
\usepackage{amssymb}
\usepackage{euscript}
\usepackage{eurosym}
\usepackage{graphicx}
\graphicspath{{img/}}
\usepackage{color}
\usepackage{epsfig}
\usepackage{fullpage}
\usepackage[colorlinks=true]{hyperref}
\usepackage{titling}

% title style
\pretitle{\noindent\LARGE}
\posttitle{\\[1ex]}
\preauthor{\large}
\postauthor{,}
\predate{\large(}
\postdate{)}
\date{last update: \today}

% large page size
\oddsidemargin -1cm
\topmargin -1cm
\textwidth 18cm
\textheight 27cm
\pagestyle{empty}


% PDF figure (floating)
\newcommand{\image}[3]{
\begin{figure}[#1]
\begin{center}
\caption{\small#3}
\includegraphics{img_#2.pdf}
\label{image:#2}
\end{center}
\end{figure}
}

% simple image
\newcommand\simpleimage[3][\linewidth]{
\smallskip\begin{center}\vbox{\noindent
\includegraphics[width=#1]{#2}\\%
\parbox{#1}{\it#3}}\end{center}
\medskip}

%\newcommand\simpleimage[3][\linewidth]{
%\smallskip\vbox{\noindent
%\includegraphics[width=#1]{#2}\\%
%\it#3}
%\medskip}


% companies
\def\SchaefferAG{\href{https://www.schaeffer-ag.de/en/}{Schaeffer-AG}}
\def\MultiCB{\href{https://portal.multi-circuit-boards.eu}{Multi-CB}}
\def\BlueFors{\href{https://www.bluefors.com/}{BlueFors}}

\def\Ohmite{\href{https://www.ohmite.com}{Ohmite}}
\def\Firmetal{\href{http://www.firmetal.com}{Firmetal}}
\def\Bruker{\href{https://www.bruker.com}{Brucker}}

% use this to refer to Farnell and RS numbers:
\def\FarnellN#1{\href{https://fi.farnell.com/#1}{Farnell:#1}}
\def\RSN#1{\href{https://fi.rsdelivers.com/productlist/search?query=#1}{RS:#1}}

% products:
\def\OhmiteOFOD{\href{https://www.ohmite.com/assets/docs/res_od_of_oa.pdf}{Ohmite OF/OD series}}

% Supplementary materials on github <report> <file>
\def\GitFile#1#2{\href{https://github.com/slazav/he3notes/raw/master/#1/#2}{#1/#2}}

% Supplementary materials on my users.aalt.fi page
\def\WWWFile#1{\href{https://users.aalto.fi/~zavyalv1/#1}{#1}}


%progams
\def\MagnettiProg{\href{https://github.com/slazav/magneetti}{\tt magnetti}}



\title{Rebuilding heat switch and nuclear stage plate for DryDemag}
\author{V.Zavjalov}

%\twocolumn
\begin{document}
\maketitle

The goal is to make a detachable frame with the nuclear demagnetization
stage (NS) and the heat switch (HS). We try to get more rigid
construction, more space for experiment cells, better heat switch.

\subsection*{Old construction}

In the previous setup NS is mounted on a brass plate (which can be bad
because of large heat capacitance and small heat conductivity of brass).
The plate is hanging on four ceramic legs under the mixing chamber of
DryDemag refregirator~\cite{drydemag}.

\image{h}{old_setup}{}
%\begin{center}
%\includegraphics[width=\linewidth]
%\end{center}

Drawings:
\begin{itemize}
\item Assembly drawing (as on the figure above):\\
\GitFile{20181105-hs}{old\_nsframe\_assembly.pdf}

\item Old brass NS flange drawing:\\
\GitFile{20181105-hs}{old\_flange\_brass.pdf}

\item Old stage flange drawing:\\
\GitFile{20181105-hs}{old\_flange\_brass.pdf}

\end{itemize}




\subsection*{Old HS performance measurements}

\subsection*{HS papers}

\subsection*{Parts}


\begin{thebibliography}{}

\bibitem{drydemag}
I. Todoshchenko, J.-P. Kaikkonen, R. Blaauwgeers, P. J. Hakonen, A. Savin,
Dry demagnetization cryostat for sub-millikelvin helium experiments: Refrigeration and thermometry,
{\it Rev.Sci.Instr}, {\bf 85}, 085106 (2014),\\
\url{https://doi.org/10.1063/1.4891619}


\bibitem{nsframe_fpd}
NS frame (top, middle, and NS plates), FPD files used for ordering parts in Schaefer-AG
\url{nsframe_fpd.zip}

\end{thebibliography}
\end{document}
