\documentclass[a4paper]{article}
% standard packages
\usepackage{amssymb}
\usepackage{euscript}
\usepackage{eurosym}
\usepackage{graphicx}
\graphicspath{{img/}}
\usepackage{color}
\usepackage{epsfig}
\usepackage{fullpage}
\usepackage[colorlinks=true]{hyperref}
\usepackage{titling}

% title style
\pretitle{\noindent\LARGE}
\posttitle{\\[1ex]}
\preauthor{\large}
\postauthor{,}
\predate{\large(}
\postdate{)}
\date{last update: \today}

% large page size
\oddsidemargin -1cm
\topmargin -1cm
\textwidth 18cm
\textheight 27cm
\pagestyle{empty}


% PDF figure (floating)
\newcommand{\image}[3]{
\begin{figure}[#1]
\begin{center}
\caption{\small#3}
\includegraphics{img_#2.pdf}
\label{image:#2}
\end{center}
\end{figure}
}

% simple image
\newcommand\simpleimage[3][\linewidth]{
\smallskip\begin{center}\vbox{\noindent
\includegraphics[width=#1]{#2}\\%
\parbox{#1}{\it#3}}\end{center}
\medskip}

%\newcommand\simpleimage[3][\linewidth]{
%\smallskip\vbox{\noindent
%\includegraphics[width=#1]{#2}\\%
%\it#3}
%\medskip}


% companies
\def\SchaefferAG{\href{https://www.schaeffer-ag.de/en/}{Schaeffer-AG}}
\def\MultiCB{\href{https://portal.multi-circuit-boards.eu}{Multi-CB}}
\def\BlueFors{\href{https://www.bluefors.com/}{BlueFors}}

\def\Ohmite{\href{https://www.ohmite.com}{Ohmite}}
\def\Firmetal{\href{http://www.firmetal.com}{Firmetal}}
\def\Bruker{\href{https://www.bruker.com}{Brucker}}

% use this to refer to Farnell and RS numbers:
\def\FarnellN#1{\href{https://fi.farnell.com/#1}{Farnell:#1}}
\def\RSN#1{\href{https://fi.rsdelivers.com/productlist/search?query=#1}{RS:#1}}

% products:
\def\OhmiteOFOD{\href{https://www.ohmite.com/assets/docs/res_od_of_oa.pdf}{Ohmite OF/OD series}}

% Supplementary materials on github <report> <file>
\def\GitFile#1#2{\href{https://github.com/slazav/he3notes/raw/master/#1/#2}{#1/#2}}

% Supplementary materials on my users.aalt.fi page
\def\WWWFile#1{\href{https://users.aalto.fi/~zavyalv1/#1}{#1}}


%progams
\def\MagnettiProg{\href{https://github.com/slazav/magneetti}{\tt magnetti}}



\title{NMR cell N1 (2016)}
\author{V.Zavjalov}

\twocolumn
\begin{document}
\maketitle

A simple cell for testing NMR in superfluid $^3$He. A Stycast-1266 cylindrical cell,
diameter 7.8~mm, height 10~mm, filling line 1~mm.

All drawings: \WWWFile{notes/2016-nmrcell1/draw.zip}\\
NMR magnet text: \WWWFile{notes/2016-nmrmag.pdf}

\simpleimage{img/cell_plan.png}{RF coil former (left) and the cell (right)}

\simpleimage{img/cell_parts.jpg}{Cell parts}

Copper cell leg was annealed at $900^\circ$C for 2h, brass spacer was
hard-soldered. Cleaning have been done in boiling water (to remove flux),
and then in boiling water with vinegar and baking soda to remove oxide. Note,
that brass became a copper-colored after soldering because Zn goes away.

Small quartz fork (32kHz) have been put into the cell leg about 1 cm from
its bottom. Cell leg have been glued into the cell with Stycast-1266.
Wires from the fork go through the glue.

Fork measurement at room temperature:\\
$P=1$~bar: $1/\tau=17.22$~1/s, $f=32743.76$~Hz\\
$P=0$~bar: $1/\tau=1.364$~1/s, $f=32750.69$~Hz

\simpleimage{img/cell_glue.jpg}{Cell leg (hard-soldered); fork (can is opened); gluing the cell}

Two pairs of RF coils are mounted on a separate plastic holder. Each pair has 34+34 turns
of 70mkm(?) Cu wire. Copper wires are in thermal contact with the mixing chamber.
Measured inductance, capacitance and resistance (with ~30cm wires):\\
coil 1: $L = 55.6\,\mu$H, $C = 14.2$~pF, $R = 14.4\,\Omega$\\
coil 2: $L = 55.4\,\mu$H,  $C = 15.7$~pF,  $R = 14.6\,\Omega$\\
Mutual capacitance about 10~pF.

Calculated magnetic field in the center: 16.6~G/A.
Calculation script (uses \MagnettiProg{} program) and field profile (text table):
\WWWFile{notes/2016-nmrcell1/rfcoil\_calc.zip}


\simpleimage{img/rfcoil_parts.jpg}{Parts of RF coil former}

\simpleimage{img/rfcoil.jpg}{RF coil (final version is on the right)}

\simpleimage{img/cell_finished.jpg}{Cell is mounted on the flange; centering RF coils}

\simpleimage{img/cell_magn.jpg}{The cell and the magnet}

RF coil forms a resonant circuit with a capacitor. A relay box (todo: text)
is used to switch the capacitance from 400 to 2400~pF.

Resonance curves measured at room temperature with a generator and a
lock-in, matlab script for fitting (not good):
\WWWFile{notes/2016-nmrmag/res300K.zip}. At frequency 800~kHz mutual inductance
and parasitic capacitance between coils compensates each other and
coupling goes to zero. This is an ideal situation for such
NMR coil design.

\simpleimage{img/resonance.png}{Resonances at room temperature}

Noise measured at room temperature:\\
\WWWFile{notes/2016-nmrmag/noise300K.zip}

\simpleimage{img/noise300K.png}{Coil noise at room temperature}

{\bf Drilling the cell (2018-01-18).} During glueing the cell filling line
was blocked. The filling line was drilled through the top wall of the cell,
then the hole was glued with a small GRP part.

\simpleimage[3cm]{img/drilling.jpg}{Drilling the cell}

{\bf RF-field calibration in the HPD experiment.} For $U_{\mbox{exc}}=1$~V
voltage on the coil is 0.05~V (HiZ mode of the generator + 20~dB
attenuator). Assuming the inductance~$L=55\,\mu$H and the
frequency~$f=1.120$~MHz one have the current~$I=0.13$~mA and field
$2.14$~mG/V in the center. Dimensionless field is $6.2\times10^{-6}$.

\end{document}
