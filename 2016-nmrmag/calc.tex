\documentclass[a4paper]{article}

% standard packages
\usepackage{amssymb}
\usepackage{euscript}
\usepackage{eurosym}
\usepackage{graphicx}
\graphicspath{{img/}}
\usepackage{color}
\usepackage{epsfig}
\usepackage{fullpage}
\usepackage[colorlinks=true]{hyperref}
\usepackage{titling}

% title style
\pretitle{\noindent\LARGE}
\posttitle{\\[1ex]}
\preauthor{\large}
\postauthor{,}
\predate{\large(}
\postdate{)}
\date{last update: \today}

% large page size
\oddsidemargin -1cm
\topmargin -1cm
\textwidth 18cm
\textheight 27cm
\pagestyle{empty}


% PDF figure (floating)
\newcommand{\image}[3]{
\begin{figure}[#1]
\begin{center}
\caption{\small#3}
\includegraphics{img_#2.pdf}
\label{image:#2}
\end{center}
\end{figure}
}

% simple image
\newcommand\simpleimage[3][\linewidth]{
\smallskip\begin{center}\vbox{\noindent
\includegraphics[width=#1]{#2}\\%
\parbox{#1}{\it#3}}\end{center}
\medskip}

%\newcommand\simpleimage[3][\linewidth]{
%\smallskip\vbox{\noindent
%\includegraphics[width=#1]{#2}\\%
%\it#3}
%\medskip}


% companies
\def\SchaefferAG{\href{https://www.schaeffer-ag.de/en/}{Schaeffer-AG}}
\def\MultiCB{\href{https://portal.multi-circuit-boards.eu}{Multi-CB}}
\def\BlueFors{\href{https://www.bluefors.com/}{BlueFors}}

\def\Ohmite{\href{https://www.ohmite.com}{Ohmite}}
\def\Firmetal{\href{http://www.firmetal.com}{Firmetal}}
\def\Bruker{\href{https://www.bruker.com}{Brucker}}

% use this to refer to Farnell and RS numbers:
\def\FarnellN#1{\href{https://fi.farnell.com/#1}{Farnell:#1}}
\def\RSN#1{\href{https://fi.rsdelivers.com/productlist/search?query=#1}{RS:#1}}

% products:
\def\OhmiteOFOD{\href{https://www.ohmite.com/assets/docs/res_od_of_oa.pdf}{Ohmite OF/OD series}}

% Supplementary materials on github <report> <file>
\def\GitFile#1#2{\href{https://github.com/slazav/he3notes/raw/master/#1/#2}{#1/#2}}

% Supplementary materials on my users.aalt.fi page
\def\WWWFile#1{\href{https://users.aalto.fi/~zavyalv1/#1}{#1}}


%progams
\def\MagnettiProg{\href{https://github.com/slazav/magneetti}{\tt magnetti}}



\title{NMR magnet optimization}
\author{V.Zavjalov}

\begin{document}

\maketitle

The field is calculated in radial coordinates,~$r$ and~$z$.
We need to optimize at least three coil systems in the volume $V_0$: $r<r_0, |z|<z_0$:
\begin{itemize}
\item Homogeneous field produced by main solenoid and compensation coils
\item Gradient field
\item Minimum (quadratic) field
\end{itemize}
For this we calculate mean square difference from the best
constant, linear, or quadratic approximation.

$$ R = \sqrt{\frac{1}{V_0}\int_V(H(r,z) - H_0 - A f(z,r))^2\,dV}
$$

Function $f(z,r)$ can be chosen according with the fitting model:
$f(z,r)=0$ for constant field, $f(z,r)=z$ for gradient field, 
$f(z,r)=z^2-r^2/2$ for quadratic field or even $f(z,r) = z^4 - 3z^2r^2 + 3/8r^2$
for forth-order field.

The $-1/2$ ratio of~$r$ and~$z$ quadratic terms comes from Maxwell equations
for a cylindrically symmetric system (see below).


This can be rewritten as
$$ R^2
 = I_2 - 2H_0 I_1 + H_0^2 + A( A I_{f2} - 2I_{fH} + 2H_0 I_{f1})
$$



where we introduced integrals
$$
I_1 = \frac{1}{V_0}\int_V H(z,r)\, dV,\qquad
I_2 = \frac{1}{V_0}\int_V H(z,r)^2\, dV,\qquad
$$
$$
I_{fH} = \frac{1}{V_0}\int_V f(z,r) H(z,r)\, dV,\qquad
I_{f1} = \frac{1}{V_0}\int_V f(z,r)  \, dV,\qquad
I_{f2} = \frac{1}{V_0}\int_V f(z,r)^2\, dV,\qquad
$$

Parameters $H_0$ and $A$ correspond the minimum of~$R$:
$$
\frac{\partial R}{\partial H_0} = 0,\quad
\frac{\partial R}{\partial A} = 0
$$

$$
H_0 = I_1 - A I_{f1},\qquad
$$
$$
A = (I_{fH} - I_{f1} I_1)/(I_{f2} - I_{f1}^2),\qquad
$$

For coil optimizations we use dimensionless current-independent
parameters:
$R/|H_0|$ for constant field, $R/(z_0|A|)$ for gradient field, and
$R/(z_0^2|A|)$ for quadratic fields.

Optimization is done in $z0=r0=10$ cylindrical volume.

{\bf Main coil geometry.} We use main coil with length 104~mm, inner radius
21~mm, and 6 layers of wire in the cylindrical shield with length 140~mm
and radius 29~mm. Wire diameter is 0.084mm (in insulation).

{\bf Compensation coils.} Minimum value in the optimization is not so
important as the minimum width ($0.2$~mm error will break all the
improvements). It's better to start compensation coils from edge of the
main solenoid and make them thin (I use 2 layers) and long -- then the
minimum is wider. Shorter coils have better field homogeneity, but it is
harder to make accurate size. Second layer with different length does not
help much. Compensation coils inside the solenoid is slightly better, but
i decided to put them outside to adjust size after making and
recalculating the main solenoid. Also it is easier to wind the main solenoid
on a flat surface. The quadratic coil can improve the field
homogeneity even more.

{\bf Gradient coils} are better if they are wider. It is good to start
them from the edge of the solenoid. Starting from the end of compensation
coils is also fine (\%10 worse). Second layer with different length can
improve gradient accuracy $\approx 2$ times,

{\bf Quadratic coil} just a coil in the center with a certain width.

{\bf Optimization results:}\\
\medskip
\begin{tabular}{lllllll}
Coils & Edge &Length       & Layers & Err 20x20$^{(1)}$  & Err 10x10$^{(2)}$        & Field [G/A] vs r,z [cm] \\\hline
Main  & 52   & 104         & 6      & $6.6\cdot10^{-4}$ & $1.7\cdot10^{-4}$ & $397.086$\\
Comp  & 52   & 11.2        & 2      & $1.4\cdot10^{-5}$ & $2.2\cdot10^{-6}$ & $397.996$\\
Grad  & 50   & 41.8        & 1      & $1.5\cdot10^{-2}$ & $3.5\cdot10^{-3}$ & $31.4270\ z$\\
Quad  & -    & 34.8        & 2      & $4.9\cdot10^{-3}$ & $3.0\cdot10^{-4}$ & $105.958 - 15.240 (z^2-r^2/2)$\\
4-th  & 13.4 & 6.4         & 2      & $1.8\cdot10^{-1}$ & $2.2\cdot10^{-1}$ & $16.6512 - 1.0351 f_4$$^{(3)}$\\
\end{tabular}
\medskip

\noindent
(1) in the volume with length = diameter = 20 mm.\\
(2) in the volume with length = diameter = 10 mm.\\
(3) $f_4 = z^4 - 3 z^2 r^2 + 3/8 r^4$


\section*{Field distribution in a cylindrical simmetry (why $z^2+r^2/2$)}

Maxwell equations:
$$
\mbox{div} B = 0,\qquad
\mbox{rot} B = 0
$$

In symmetrical case ($\partial B/\partial\phi = 0, B_\phi=0$):
$$
\frac{\partial B_r}{\partial z} = \frac{\partial B_z}{\partial r}, \qquad
\frac{1}{r} \frac{\partial r B_r}{\partial r} + \frac{\partial B_z}{\partial z}=0.
$$

Multiply the first equation by $r$, differentiate by $r$, and then multiply by $1/r$.
Differentiane the second equation by $z$. Then one can exclude $B_r$:
$$
\frac{1}{r} \frac{\partial B_z}{\partial r}
 + \frac{\partial^2 B_z}{\partial r^2}
 + \frac{\partial^2 B_z}{\partial z^2}=0.
$$

For quadratic distribution $B_z = az^2 + br^2$ one have $a=-2b$.

For 4-power distribution $B_z \propto z^4 - 3 z^2 r^2 + 3/8 r^4$.

\section*{Critical current}

For calculated solenoid (main + compensation coils), maximum absolute
field on the coil is 0.110~T/A, on the shield is 0.843~T/A.
According with Bruker wire specification critical field is $(13-5I)$~T.
Critical field for Nb shield as about 0.15~T.

Thus the current limit introduced by shield is 1.35~A,
current limit by wire is 2.57~A. This corresponds to the
field in the middle 0.09~T and 1.71~T (???).

\end{document}
